%%%%%%%%%%%%%%%%%%%%%%%%%%%%%%%%%%%%%%%%%%%%%%%%%%
\newcommand{\suppmaterial}{Anciliary files}
\newcommand{\onlineappendices}{Online Appendices (\href{http://compstorylab.org/trumpstoryturbulence/}{compstorylab.org/trumpstoryturbulence/})}
\newcommand{\onlineappendicesplain}{Online Appendices}
%% \newcommand{\suppmaterial}{Supplementary Material}
%% \newcommand{\suppmaterial}{Appendices}
\newcommand{\arxivonly}[1]{#1}


%% allow hyphenation for urls with long many-hyphenated strings
\PassOptionsToPackage{hyphens}{url}\usepackage{hyperref}
\makeatletter
\g@addto@macro{\UrlBreaks}{\UrlOrds}
\makeatother

%% remove default typewriter nastiness for urls
\urlstyle{same}
\hypersetup{
  colorlinks=true,
  allcolors=todoblue,
  urlcolor=todoblue,
  citecolor=todoblue,
  pdfborder={0 0 0},
  breaklinks=true,
}

\usepackage{amssymb,amsmath}
\usepackage{color}
\definecolor{todoblue}{RGB}{0, 91, 187}
\newcommand{\todo}[1]{\noindent\textcolor{todoblue}{{$\Box$ #1}}}

%%%%%%%%%%%%%%%%%%%%%%%%%%%%%%%%%%%%%%%%%%%%%
%% layout
%%%%%%%%%%%%%%%%%%%%%%%%%%%%%%%%%%%%%%%%%%%%%
\usepackage[margin=1.0in]{geometry}
\setlength{\parindent}{0pt}
\setlength{\parskip}{5pt}

%%%%%%%%%%%%%%%%%%%%%%%%%%%%%%%%%%%%%%%%%%%%%
%% general packages
%%%%%%%%%%%%%%%%%%%%%%%%%%%%%%%%%%%%%%%%%%%%%

\PassOptionsToPackage{svgnames,table}{xcolor}
\usepackage{tikz}
\usetikzlibrary{shapes}
\usepackage{amsthm,amsmath}
\usepackage[normalem]{ulem}
\usepackage[makeroom]{cancel}
\usepackage{titlesec}
\titleformat{\section}[block]{\color{black}\Large\bfseries\filcenter}{}{1em}{}

\usepackage{graphics}
\usepackage{rotating}
\usepackage{array}
\usepackage{amsmath}
\usepackage{textcomp}
\usepackage{hyperref}
\usepackage{listings}

%%%%%%%%%%%%%%%%%%%%%%%%%%%%%%%%%%%%%%%%%%%%%
%% colors
%%%%%%%%%%%%%%%%%%%%%%%%%%%%%%%%%%%%%%%%%%%%%

\usepackage{color}

\definecolor{lightgrey}{rgb}{0.7,0.7,0.7}
\definecolor{grey}{rgb}{0.5,0.5,0.5}
\definecolor{lightblue}{RGB}{50, 90, 187}
\definecolor{darkgrey}{rgb}{0.3,0.3,0.3}

\tikzstyle{mybox} = [draw=darkgrey, fill=grey!10, very thick,
    rectangle, rounded corners, inner sep=10pt, inner ysep=20pt]
\tikzstyle{fancytitle} =[fill=lightblue!70, text=white]

%%%%%%%%%%%%%%%%%%%%%%%%%%%%%%%%%%%%%%%%%%%%%
%% structures for general (e.g., editor) and reviewer comments
%%
%% reply is in normal format below each comment
%%%%%%%%%%%%%%%%%%%%%%%%%%%%%%%%%%%%%%%%%%%%%

\newcounter{reviewer}
\newcounter{comment}

\newcommand{\reviewerheader}[1]{
  \subsection*{Response to Reviewer \##1:}
  \renewcommand*{\thereviewer}{#1}
  \setcounter{comment}{1}
}

\newcommand{\generalcomment}[2]{
\bigskip
\begin{tikzpicture}
  \node [mybox] (box){%%
    \begin{minipage}{0.95\textwidth}
      #2 %% comment
    \end{minipage}
  };
  %% #1 = any heading
  \node[fancytitle, right=8pt] at (box.north west) {#1}; 
\end{tikzpicture}%

}


\newcommand{\reviewercomment}[1]{
  \bigskip
  \begin{tikzpicture}
    \node [mybox] (box){%
      \begin{minipage}{0.95\textwidth}
        #1 %% comment
      \end{minipage}
    };
    %% use counters
    \node[fancytitle, right=8pt] at (box.north west) {Comment \thecomment\ by Reviewer \thereviewer};
  \end{tikzpicture}%

  \addtocounter{comment}{1}
}

%% \newcommand{\reply}[1]{\vspace{5mm}#1}


